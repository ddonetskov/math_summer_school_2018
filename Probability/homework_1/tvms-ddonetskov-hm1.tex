% !TEX TS-program = lualatex

% required for TexXstudio, an alternative is to change the default bibliography tool 
% in TeXstudio settings (Options > Configure TeXstudio > Build > Default Bibliography Tool)
% !BIB TS-program = biber

\documentclass[a4paper,11pt]{article}

\usepackage{geometry}
\geometry{left=20.00mm, right=15.00mm, top=20.00mm, bottom=20.00mm}

% ------------------------------------------------------------------------------
% Title
% ------------------------------------------------------------------------------

\title{Теория вероятностей и математическая статистика \\
Домашнее задание №1}
\author{Дмитрий Донецков (ddonetskov@gmail.com)}
\date{\today}

% ------------------------------------------------------------------------------
% LUA
% ------------------------------------------------------------------------------

% \usepackage{luacode}

% ------------------------------------------------------------------------------
% Graphics
% ------------------------------------------------------------------------------

\usepackage{graphicx}

% ------------------------------------------------------------------------------
% Figures
% ------------------------------------------------------------------------------

\usepackage{subcaption}         % subcaptions in figures

% ------------------------------------------------------------------------------
% Tables
% ------------------------------------------------------------------------------

\usepackage{multirow}           % spanning columns across multiple rows
\usepackage{makecell}           % allows different formats inside cells

\renewcommand\theadalign{bc}
\renewcommand\theadfont{\bfseries}
\renewcommand\theadgape{\Gape[4pt]}
\renewcommand\cellgape{\Gape[4pt]}

% ------------------------------------------------------------------------------
% Math (Additional Support)
% ------------------------------------------------------------------------------

\usepackage{amsmath,amsfonts,amssymb,amsthm,mathtools}     % AMS
\usepackage{cancel}             % four different modes of striking through
\usepackage{dsfont}
\usepackage{icomma}             % Smart comma: $0,2$ --- число, $0, 2$ --- перечисление
\usepackage{nicefrac}

\DeclareMathOperator*{\E}{\mathbb{E}}   % the expectation symbol
\DeclareMathOperator*{\D}{\mathbb{D}}   % the dispersion symbol

% ------------------------------------------------------------------------------
% Russian Language (support thereof)
% ------------------------------------------------------------------------------

\usepackage{csquotes}
\usepackage[english,russian]{babel}	    % локализация и переносы

% ------------------------------------------------------------------------------
% Fonts 
% ------------------------------------------------------------------------------

\usepackage{fontspec}           % required to load Open Type, True Type fonts
\setmainfont{Linux Libertine O} % Libertine covers Latin, Hebrew, Greek, and Russian

\usepackage{euscript}	          % Шрифт Евклид
\usepackage{mathrsfs}           % Красивый матшрифт
\setmonofont{Courier New}

% ------------------------------------------------------------------------------
% Bookmarking, citing, URL's
% ------------------------------------------------------------------------------

% hyperref usually needs to be loaded last
\usepackage{hyperref}
\usepackage{url}
\usepackage[dvipsnames]{xcolor}

\hypersetup{
    colorlinks=true,
    linkcolor=blue,
    filecolor=red,      
    urlcolor=blue,
}

\urlstyle{same}

\begin{document}

\maketitle

\section{Вопросы}

\subsection{Вопрос 1}

Такое событие можно трактовать и следующим образом: события A, B, C не должны происходить одновременно. Тогда, искомое событие: $\overline{A \cap B \cap C } = \overline{A} \cup \overline{B} \cup \overline{C}$.

\subsection{Вопрос 2}

Да. Формула условной вероятности даёт вероятность при условии наличия другого события, при этом обладая всеми теми же свойствами, что и простая вероятность. Например, если событие B никогда не случается при наличии события A, то $P(B|A) = 0$, и, наоборот, если событие B всегда случается при наличии события A, то $P(B|A) = 1$. На диаграмме Эйлера первый случай может быть представлен непересекающимися областями, а второй - областью для B, лежащей полностью внутри области для A.

\subsection{Вопрос 3}

Равно 1. Согласно формуле полной вероятности, события $A_i$ разбивают пространство исходов таким образом, что для всех $i$: $\bigcap_{i}{A_i} = \emptyset$, $\bigcup_{i}{A_i} = \Omega$. Соответственно, сумма данных событий (гипотез) равна 1.

\subsection{Вопрос 4}

\begin{equation*}
\begin{split}
\D[X+Y]\rvert_{\E[XY]=\E[X]\E[Y]} & = \E[(X+Y)^2] - (\E[X+Y])^2 \\
& = \E[X^2 + 2XY + Y^2] - (\E[X] + \E[Y])^2 \\
& = \E[X^2 + 2XY + Y^2] - (\E[X])^2 - 2\E[X][Y] - (\E[Y])^2 \\
& = \E[X^2] + 2\E[XY] + \E[Y^2] - (\E[X])^2 - 2\E[X][Y] - (\E[Y])^2 \\
& = \E[X^2] + \cancel{2\E[X]\E[Y]} + \E[Y^2] - (\E[X])^2 - \cancel{2\E[X][Y]} - (\E[Y])^2 \\
& = \E[X^2] + \E[Y^2] - (\E[X])^2 - (\E[Y])^2 \\
& = (\E[X^2] - (\E[X])^2) + (\E[Y^2] - (\E[Y])^2) \\
& = \D[X] + \D[Y].
\end{split}
\end{equation*}

\section{Задачи}

\subsection{Задача 1}

Искомые элементарные исходы пространства $\Omega$ при заданном эксперименте - это множество последовательностей вида РР...РРГ, где каждый символ обозначает, что выпало при $i$-м подбрасывании монеты: Р - выпадение решки на $i$-м шаге, Г - выпадение герба, $i \in [0,n]$. Символ "Г" всегда стоит единственным и последним в данной последовательности, т.к. выпадение герба означает завершение эксперимента.

\subsection{Задача 2}

Обозначим $X$ - сумма значений двух игральных костей.

X принимает значения от 2 до 12 с вероятностями, соответствующими отношению частоты выпадения суммы (во всех возможных комбинациях) к количеству всех возможных комбинаций, коих 36. Тогда, получается следующее распределение:

\medskip

\begin{tabular}{|c|c|c|c|c|c|c|c|c|c|c|c|c|}
\hline 
X & 2 & 3 & 4 & 5 & 6 & 7 & 8 & 9 & 10 & 11 & 12 \\ 
\hline 
\shortstack{Возможные \\ комбинации \\ сумм} & 
  1+1 & 
  \shortstack{ 1+2 \\ 2+1} & 
  \shortstack{ 1+3 \\ 2+2 \\ 3+1} & 
  \shortstack{ 1+4 \\ 2+3 \\ 3+2 \\ 4+1} & 
  \shortstack{ 1+5 \\ 2+4 \\ 3+3 \\ 4+2 \\ 5+1} & 
  \shortstack{ 1+6 \\ 2+5 \\ 3+4 \\ 4+3 \\ 5+2 \\ 6+1} &  
  \shortstack{ 2+6 \\ 3+5 \\ 4+4 \\ 5+3 \\ 6+2 } &  
  \shortstack{ 3+6 \\ 4+5 \\ 5+4 \\ 6+3 } & 
  \shortstack{ 4+6 \\ 5+5 \\ 6+4 } &
  \shortstack{ 5+6 \\ 6+5 } & 
  \shortstack{ 6+6 } \\ 
\hline 
\shortstack{Количество \\ комбинаций} & 1 & 2 & 3 & 4 & 5 & 6 & 5 & 4 & 3 & 2 & 1 \\ 
\hline 
\shortstack{Вероятность} & $\nicefrac{1}{36}$ & $\nicefrac{2}{36}$ & $\nicefrac{3}{36}$ & $\nicefrac{4}{36}$ & $\nicefrac{5}{36}$ & $\nicefrac{6}{36}$ & $\nicefrac{5}{36}$ & $\nicefrac{4}{36}$ & $\nicefrac{3}{36}$ & $\nicefrac{2}{36}$ & $\nicefrac{1}{36}$ \\ 
\hline 
\end{tabular}

\bigskip

Математическое ожидание $\E[X]$:

\begin{multline*}
\E[X] = \sum_{i: x_i\in[2,12]}{P(X=x_i) \times x_i} = \\
 2 \times \frac{1}{36} +  3 \times \frac{2}{36} + 4  \times \frac{3}{36} + 
 5 \times \frac{4}{36} +  6 \times \frac{5}{36} + 7  \times \frac{6}{36} +
 8 \times \frac{5}{36} +  9 \times \frac{4}{36} + 10 \times \frac{3}{36} +
11 \times \frac{2}{36} + 12 \times \frac{1}{36} \\
 = \frac{252}{36} = 7.
\end{multline*}

\subsection{Задача 3}

Необходимо вычислить все значения $Y$ для заданных значений $X$, вероятности остаются теми же, т.к. зависимость $Y$ и $X$ задана тригонометрической функцией, не вероятностной.

\medskip

\begin{tabular}{|c|c|c|c|c|c|}
\hline 
X & -2 & -1 & 0 & 1 & 2 \\ 
\hline 
p & 0 & \nicefrac{1}{4} & \nicefrac{1}{2} & \nicefrac{1}{8} & \nicefrac{1}{8} \\ 
\hline 
Y & 3 & 3 & 3 & 3 & 3 \\
\hline 
\end{tabular} 

\bigskip

Математическое ожидание: $\E[Y] = 3 \times 1 = 3$, дисперсия: $\D[Y] = \E[Y^2] - (E[Y])^2 = 0$.

\subsection{Задача 4}

В колоде из 52 карт находится 13 бубновых карт. Для того, чтобы посчитать вероятность выбора хотя бы двух бубновых карт при выборе шести случайных карт из колоды, то необходимо посчитать 

\begin{itemize}
\item $n_1$ - количество всех возможных комбинаций из шести случайно выбранных карт, в которых бубновых карт две и больше,
\item $n_2$ - количество всех возможных комбинаций выбора шести произвольных карт из колоды.
\end{itemize}

Значение $n_1$ можно также подсчитать как разницу между $n_2$ и количеством всех возможных комбинаций из шести случайно выбранных карт, в которых бубновых карт меньше двух. Такой подход поможет прийти к тому же результату при меньшем количестве действий.

Отношение $n_1$ к $n_2$ и даст искомую вероятность.

Количество всех возможных комбинаций выбора шести произвольных карт из колоды:

\[n_2 = C^6_{52} = \frac{52!}{46!6!} = 20358520.\]

Количество всех возможных комбинаций из шести случайно выбранных карт, в которых бубновых карт меньше двух - это сумма двух комбинаций: шести карт из всех небубновых или пяти карт из всех небубновых и одной карты их любых бубновых:

\begin{equation*}
\begin{split}
C^{6}_{39} + C^{5}_{39} C^{1}_{13} & = \frac{39!}{33!6!} + \frac{39!}{34!5!} \frac{13!}{12!1!} \\
& = \frac{39 \times 38 \times 37 \times 36 \times 35 \times 34 }{6!} +
  \frac{39 \times 38 \times 37 \times 36 \times 35}{5!} \times 13 \\
& = 10747464.
\end{split}
\end{equation*}

Тогда

\begin{equation*}
\begin{split}
p = \frac{n_1}{n_2} & = \frac{C^6_{52} - (C^{6}_{39} + C^{5}_{39} C^{1}_{13})}{C^6_{52}} 
  = 1 - \frac{C^{6}_{39} + C^{5}_{39} C^{1}_{13}}{C^6_{52}} = 1 - \frac{10747464}{20358520}
\approx 0.47.
\end{split}
\end{equation*}

\subsection{Cложная задача}

Сложность задачи обусловлена тем, что буквы в последовательности могут появляться с разной вероятностью. Количество возможных комбинаций слова - $4^{15}$. Каждая из этих комбинаций будет иметь свою собственную вероятность появления, мы не можем сделать допущение, что они будут равновероятны.

(не завершена)

% ------------------------------------------------------------------------------
% Bibliography
% Is that possible to make it as per ГОСТ Р7.05-2008?
% ------------------------------------------------------------------------------
%\printbibliography[heading=bibintoc,title={References}]

\end{document}
