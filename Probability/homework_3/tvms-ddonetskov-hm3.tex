% !TEX TS-program = lualatex

% required for TexXstudio, an alternative is to change the default bibliography tool 
% in TeXstudio settings (Options > Configure TeXstudio > Build > Default Bibliography Tool)
% !BIB TS-program = biber

\documentclass[a4paper,11pt]{article}

\usepackage{geometry}
\geometry{left=20.00mm, right=15.00mm, top=20.00mm, bottom=20.00mm}

% this is to get rid of 'Overfull \hbox...' errors
\usepackage{microtype}

% ------------------------------------------------------------------------------
% Title
% ------------------------------------------------------------------------------

\title{\vspace{-1.5cm}Теория вероятностей и математическая статистика \\
Домашнее задание №3}
\author{Дмитрий Донецков (ddonetskov@gmail.com)}
\date{\today}

% ------------------------------------------------------------------------------
% LUA
% ------------------------------------------------------------------------------

% \usepackage{luacode}

% ------------------------------------------------------------------------------
% Graphics
% ------------------------------------------------------------------------------

\usepackage{graphicx}

% https://www.sharelatex.com/learn/Pgfplots_package
\usepackage{pgfplots}
\pgfplotsset{compat=1.15}\usepgfplotslibrary{fillbetween}

\usepackage{wrapfig}

% ------------------------------------------------------------------------------
% Figures
% ------------------------------------------------------------------------------

\usepackage{subcaption}         % subcaptions in figures

% ------------------------------------------------------------------------------
% Tables
% ------------------------------------------------------------------------------

\usepackage{multirow}           % spanning columns across multiple rows
\usepackage{makecell}           % allows different formats inside cells

\renewcommand\theadalign{bc}
\renewcommand\theadfont{\bfseries}
\renewcommand\theadgape{\Gape[4pt]}
\renewcommand\cellgape{\Gape[4pt]}

% ------------------------------------------------------------------------------
% Math (Additional Support)
% ------------------------------------------------------------------------------

\usepackage{amsmath,amsfonts,amssymb,amsthm,mathtools}     % AMS
\usepackage{cancel}             % four different modes of striking through
\usepackage{dsfont}
\usepackage{icomma}             % Smart comma: $0,2$ --- число, $0, 2$ --- перечисление
\usepackage{nicefrac}
\usepackage{physics}            % implementation of \abs and \norm

\DeclareMathOperator*{\D}{\mathbb{D}}   % the dispersion symbol
\DeclareMathOperator*{\E}{\mathbb{E}}   % the expectation symbol
\DeclareMathOperator*{\N}{\mathbb{N}}   % the set of natural numbers
\DeclareMathOperator*{\R}{\mathbb{R}}   % the set of real numbers
\DeclareMathOperator*{\Z}{\mathbb{Z}}   % the set of integers

% ------------------------------------------------------------------------------
% Russian Language (support thereof)
% ------------------------------------------------------------------------------

\usepackage[russian,english]{babel}	    % локализация и переносы

% ------------------------------------------------------------------------------
% Fonts 
% ------------------------------------------------------------------------------

\usepackage{fontspec}           % required to load Open Type, True Type fonts

\setmainfont{CMU Serif}
\setsansfont{CMU Sans Serif}
\setmonofont{CMU Typewriter Text}

%\setmainfont{Linux Libertine O} % Libertine covers Latin, Hebrew, Greek, and Russian
%\setmonofont{Courier New}

\usepackage{euscript}	          % Шрифт Евклид
\usepackage{mathrsfs}           % Красивый матшрифт

% ------------------------------------------------------------------------------
% Bibliography 
% ------------------------------------------------------------------------------

% Removed as it was not required.

% ------------------------------------------------------------------------------
% Bookmarking, citing, URL's
% ------------------------------------------------------------------------------

% hyperref usually needs to be loaded last
\usepackage{hyperref}
\usepackage{url}
% \usepackage[dvipsnames]{xcolor}

\hypersetup{
    colorlinks=true,
    linkcolor=blue,
    filecolor=red,      
    urlcolor=blue,
}

\urlstyle{same}

\begin{document}

\maketitle

\section{Вопросы}

\subsection{Вопрос 1}

N.B. Данная задача решена с ошибкой.

Случайным величинам $X \sim U[0,2\pi]$, $Y = g(X)$ соответствуют функции распределения:

\begin{align*}
  F_X(x) &=  P(X \leq x) = \frac{x}{2\pi} \\
  F_Y(y) & = P(Y \leq y) = P(g(X) \leq y) = P(X \leq g^{-1}(y)), y \in [-1, 1].
\end{align*}

Если $g(x) = \sin x$, то $g^{-1}(y) = \asin x$. Для заданного интервала $y \in [-1, 1]$ такая обратная функция не существует в силу многозначности отображений из области определения в область значений. Но, если учитывать только главные значения функции $\asin$, с учётом того, что мера аргументов - вероятностная, то можно записать

\begin{align*}
  F_Y(y) & = 
  \begin{cases}
    0 & , y < -1 \\
    P(\pi - \asin y \leq X \leq 2\pi + \asin y) = F_X(2\pi + \asin y) - F_X(\pi - \asin y) & , \text{если}\ y \in [-1, 0] \\
    P(\asin y \leq X <     \pi - \asin y) = F_X( \pi - \asin y) - F_X(\asin y) & , \text{если}\ y \in ( 0, 1] \\
    1 & , y > 1
  \end{cases}
  \\
  & =
  \begin{cases}
    0 & , y < -1 \\
    \frac{1}{2\pi}(2\pi + \asin y - \pi + \asin y)
      = \frac{1}{2} + \frac{\asin y}{\pi} & , \text{если}\ y \in [-1, 0] \\
    \frac{1}{2\pi}(\pi - \asin y - \asin y)
      = \frac{1}{2} - \frac{\asin y}{\pi} & , \text{если}\ y \in ( 0, 1] \\
    1 & , y > 1.
  \end{cases}  
\end{align*}

В общем виде. Если известно, что $X \sim F_X(x), Y = g(X)$, то $F_Y(y)$ можно определить через $F_X(x)$ при условии, что существует обратная функция $g^{-1}(y)$. Если функция $g$ необратима в алгебраическом смысле, то следует исследовать как её можно обратить в смысле вероятностной меры. Интуитивно, такое обращение всегда можно подобрать.

\subsection{Вопрос 2}

В случае $X \sim Bin(n, p)$ можно задать такое малое $p$, что для подсчёта $P(X=k)$ использование нормального распределения (согласно теореме Муавра-Лапласа) будет \textit{малоприменимым} в силу малого значения $np$ (рекомендуется, чтобы оно было больше 5), а использование же распределения Пуассона будет \textit{применимым}, т.к. теорема Пуассона не накладывает подобных ограничений на нижний порог для $\lambda$. Так, например, для серии событий длиною $n = 1000$ с редким успешным событием вероятностью $p=0.001$: $np$ = 1, $\lambda = 1$.

Исходя из вышеизложенного, распределение Пуассона более применимо для работы с редкими событиями, чем нормальное, отсюда, видимо, и такое название.

\subsection{Вопрос 3}

Для $X \sim \mathcal{N}(\mu,\,\sigma^{2})$:

\begin{align*}
P(a \leq X \leq b) = P(\frac{a-\mu}{\sigma} \leq \frac{X - \mu}{\sigma} \leq \frac{b-\mu}{\sigma})
= \Phi_0(\frac{b-\mu}{\sigma}) - \Phi_0(\frac{a-\mu}{\sigma}).
\end{align*}

\subsection{Вопрос 4}

\begin{align*}
\begin{cases}
  P(|X-\E[X]| < a) = 1 - P(|X-\E[X]| > a), \\
  P(|X-\E[X]| > a) \leq \frac{\D[X]}{a^2}
\end{cases}
\Rightarrow
P(|X-\E[X]| < a) \geq 1 - \frac{\D[X]}{a^2}.
\end{align*}

\section{Задачи}

\subsection{Задача 1}

Обозначим за $X$ время спуска лыжника. По условию задачи $X \sim \mathcal{N}(\mu = 12.3,\,\sigma^{2} = 0.4^2)$. Тогда:

\begin{align*}
P(12.1 \leq X \leq 12.5) 
& = P(\frac{12.1-\mu}{\sigma} \leq \frac{X - \mu}{\sigma} \leq \frac{12.5-\mu}{\sigma}) \\
& = P(\frac{12.1-12.3}{0.4} \leq \frac{X - 12.3}{0.4} \leq \frac{12.5-12.3}{0.4}) \\
& = P(-0.5 \leq \frac{X - 12.3}{0.4} \leq 0.5) = \Phi_0(0.5) - \Phi_0(-0.5) \\
& = 0.69146 - 0.30853 = 0.38293.
\end{align*}

\subsection{Задача 2}

Обозначим за $X$ количество бракованных свечей. По условию задачи $X \sim Bin(p = 0.01)$.

\medskip

Для нахождения $P(X \leq 1)$ в соответствии с теоремой Пуассона, следует задаться параметром $\lambda$, который выражает частоту повторения исследуемого события. Для партии $n = 25$ и $p = 0.01$, частота $\lambda$ составит $\lambda = pn = 0.25$. Тогда,

\begin{align*}
P(X \leq 1) = P(X=0) + P(X=1) = \sum_{k=0}^{1}\frac{\lambda^k}{k!}e^{-\lambda}
= e^{-\lambda}\sum_{k=0}^{1}\frac{\lambda^k}{k!} = e^{-0.25}(1 + 0.25) \approx 0.97.
\end{align*}

\medskip

Для второго случая, где $n = 1000$, $P(X = 20)$ согласно теореме Муавра-Лапласа ($\mu = 10, \sigma = 9.9)$:

\begin{align*}
P(X = 20) & = P(19 < X < 21) = |\text{переход к нормальному распределению с учётом поправки}\ \frac{1}{2}| \\ 
& = P(19.5 \leq X \leq 20.5) = P(\frac{19.5-10}{9.9} < \frac{X-10}{9.9} < \frac{20.5-10}{9.9}) \\
& = \Phi_0(1.06) - \Phi_0(0.96) = 0.85543 - 0.83147 \approx 2.4\%.
\end{align*}

\subsection{Сложная задача}

(не выполнена)

% ------------------------------------------------------------------------------
% Bibliography
% Is that possible to make it as per ГОСТ Р7.05-2008?
% ------------------------------------------------------------------------------
%\printbibliography[heading=bibintoc,title={References}]

\end{document}
