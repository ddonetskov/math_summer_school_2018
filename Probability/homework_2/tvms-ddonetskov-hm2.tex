% !TEX TS-program = lualatex

% required for TexXstudio, an alternative is to change the default bibliography tool 
% in TeXstudio settings (Options > Configure TeXstudio > Build > Default Bibliography Tool)
% !BIB TS-program = biber

\documentclass[a4paper,11pt]{article}

\usepackage{geometry}
\geometry{left=20.00mm, right=15.00mm, top=20.00mm, bottom=20.00mm}

% this is to get rid of 'Overfull \hbox...' errors
\usepackage{microtype}

% ------------------------------------------------------------------------------
% Title
% ------------------------------------------------------------------------------

\title{\vspace{-1.5cm}Теория вероятностей и математическая статистика \\
Домашнее задание №2}
\author{Дмитрий Донецков (ddonetskov@gmail.com)}
\date{\today}

% ------------------------------------------------------------------------------
% LUA
% ------------------------------------------------------------------------------

% \usepackage{luacode}

% ------------------------------------------------------------------------------
% Graphics
% ------------------------------------------------------------------------------

\usepackage{graphicx}

% https://www.sharelatex.com/learn/Pgfplots_package
\usepackage{pgfplots}
\pgfplotsset{compat=1.15}\usepgfplotslibrary{fillbetween}

\usepackage{wrapfig}

% ------------------------------------------------------------------------------
% Figures
% ------------------------------------------------------------------------------

\usepackage{subcaption}         % subcaptions in figures

% ------------------------------------------------------------------------------
% Tables
% ------------------------------------------------------------------------------

\usepackage{multirow}           % spanning columns across multiple rows
\usepackage{makecell}           % allows different formats inside cells

\renewcommand\theadalign{bc}
\renewcommand\theadfont{\bfseries}
\renewcommand\theadgape{\Gape[4pt]}
\renewcommand\cellgape{\Gape[4pt]}

% ------------------------------------------------------------------------------
% Math (Additional Support)
% ------------------------------------------------------------------------------

\usepackage{amsmath,amsfonts,amssymb,amsthm,mathtools}     % AMS
\usepackage{cancel}             % four different modes of striking through
\usepackage{dsfont}
\usepackage{icomma}             % Smart comma: $0,2$ --- число, $0, 2$ --- перечисление
\usepackage{nicefrac}
\usepackage{physics}            % implementation of \abs and \norm

\DeclareMathOperator*{\D}{\mathbb{D}}   % the dispersion symbol
\DeclareMathOperator*{\E}{\mathbb{E}}   % the expectation symbol
\DeclareMathOperator*{\N}{\mathbb{N}}   % the set of natural numbers
\DeclareMathOperator*{\R}{\mathbb{R}}   % the set of real numbers
\DeclareMathOperator*{\Z}{\mathbb{Z}}   % the set of integers

% ------------------------------------------------------------------------------
% Russian Language (support thereof)
% ------------------------------------------------------------------------------

\usepackage[russian,english]{babel}	    % локализация и переносы

% ------------------------------------------------------------------------------
% Fonts 
% ------------------------------------------------------------------------------

\usepackage{fontspec}           % required to load Open Type, True Type fonts

\setmainfont{CMU Serif}
\setsansfont{CMU Sans Serif}
\setmonofont{CMU Typewriter Text}

%\setmainfont{Linux Libertine O} % Libertine covers Latin, Hebrew, Greek, and Russian
%\setmonofont{Courier New}

\usepackage{euscript}	          % Шрифт Евклид
\usepackage{mathrsfs}           % Красивый матшрифт

% ------------------------------------------------------------------------------
% Bibliography 
% ------------------------------------------------------------------------------

% Removed as it was not required.

% ------------------------------------------------------------------------------
% Bookmarking, citing, URL's
% ------------------------------------------------------------------------------

% hyperref usually needs to be loaded last
\usepackage{hyperref}
\usepackage{url}
% \usepackage[dvipsnames]{xcolor}

\hypersetup{
    colorlinks=true,
    linkcolor=blue,
    filecolor=red,      
    urlcolor=blue,
}

\urlstyle{same}

\begin{document}

\maketitle

\section{Вопросы}

\subsection{Вопрос 1}

Плотность распределения случайной величины определяется такой функцией (называемой функцией плотности распределения), которая удовлетворяет следующим условиям:

\begin{itemize}
\item $f(x) \geq 0, x \in \R$,
\item $\int_{-\infty}^{+\infty}f(x)dx = 1$.
\end{itemize}

\subsection{Вопрос 2}

Функция $F(x)$ не является функцией распределения, так как она не растёт монотонно на области определения, что противоречит её определению. Так, для данной функции $P(X \leq 0) > P(X \leq 1)$, что некорректно.

\subsection{Вопрос 3}

Да, могут, теория вероятностей не накладывает ограничения по уникальности распределений для случайных величин. Пример: случайные величины $X$, $Y$ - выпавшие значения кубиков 1 и 2, соответственно. У них будут одинаковые распределения (таблицы распределений), при этом это две разные случайные величины.

\section{Задачи}

\subsection{Задача 1}

Воспользуемся свойством функции плотности распределения:

\begin{align*}
\int_{-\infty}^{+\infty}f(x)dx = 1 
\Rightarrow 
\int_{1}^{+\infty}\frac{C}{x^4}dx = 1
\Rightarrow 
\lim_{x' \rightarrow +\infty} \int_{1}^{x'}Cx^{-4}dx = 1.
\end{align*}

Первообразная для подынтегральной функции $F(x) = -C\frac{1}{3}x^{-3} + 1$. Тогда

\begin{align*}
\lim_{x' \rightarrow +\infty} \int_{1}^{x'}Cx^{-4}dx = 1
& \Rightarrow 
\lim_{x' \rightarrow +\infty} F(x') - F(1) = 1
\Rightarrow 
-C\frac{1}{3} \lim_{x \rightarrow +\infty} x^{-3} - -C\frac{1}{3} = 1
\Rightarrow 
\boxed{C = 3} \\
& \Rightarrow 
f(x) = 
\begin{cases}
  0 & , \text{if}\ x < 1 \\
  3x^{-4} & , \text{if}\ x \geq 1
\end{cases}
\ \ \
F(x) = 
\begin{cases}
  0 & , \text{if}\ x < 1 \\
  1-x^{-3} & , \text{if}\ x \geq 1
\end{cases}
\end{align*}

\begin{align*}
P(X<3) = P(1 < X < 3) = \int_{1}^{3}f(x)dx = F(x)\rvert_1^3 = -x^{-3} \rvert_1^3 
= -\frac{1}{3^3} - -1 = \boxed{\frac{26}{27}}.
\end{align*}

\begin{align*}
P(X>7) = \int_{7}^{+\infty}f(x)dx = \lim_{x \Rightarrow +\infty} F(x) - F(7) = 0 - -\frac{1}{7^3} = \boxed{\frac{1}{343}}.
\end{align*}


\subsection{Задача 2}

Обозначим: 

\begin{itemize}
\item $X$ - количество страховых случаев,
\item $n$ = 40000 - количество договоров,
\item $p$ = 0.02 - вероятность страхового случая.
\end{itemize}

При заданных условиях $X \sim Bin(n, p)$ ($X$ распределена согласно биномиальному распределению).

Воспользуемся теоремой Муавра-Лапласа для подсчёта $P(X \leq 870)$, т.к. для подсчёта по биномиальным коэффициентам придётся считать очень большой ряд, а заданные параметры (большое $n$) позволяют воспользоваться теоремой. Тогда:

\begin{align*}
\E[X] & = np \Rightarrow \mu = 800, \\
\D[X] & = npq \Rightarrow \sigma^2 = 784.
\end{align*}

Искомая вероятность:

\begin{align*}
P(X \leq 870) = P(\frac{X - 800}{\sqrt{784}} \leq \frac{870 - 800}{\sqrt{784}})
= P(\frac{X - 800}{28} \leq 2.5) = F_0(2.5) = .99379.
\end{align*}

\subsection{Задача 3}

\begin{wrapfigure}{r}{0.5\textwidth}
  \begin{center}
  \begin{tikzpicture}
  \begin{axis}[
      % only scale the axis, not the axis including the ticks and labels
      scale only axis=true,
      % set `width' and `height' to the desired values
      width=6cm,
      height=6cm,
      xlabel = $\text{Время появления}\ A$,
      ylabel = $\text{Время появления}\ B$,
      xmin=0, xmax=60,
      ymin=0, ymax=60,
      xtick={0,10,20,30,40,50,60}, 
      ytick={0,10,20,30,40,50,60},
      xmajorgrids=true,
      ymajorgrids=true,
      grid style=dashed,
      axis on top
    ]
    \addplot [
        name path=L1,
        domain=0:60, 
        samples=2, 
        color=red,
        line width=1pt
    ] {x + 20};
    \addplot [
        name path=L2,
        domain=0:60, 
        samples=2, 
        color=red,
        line width=1pt
    ] {x - 20};
    \addplot [yellow!50] fill between [of=L1 and L2];
    \node at (axis cs:30,30) {$S$};
  \end{axis}
  \end{tikzpicture}
  \end{center}
  \caption{Интервалы встречи A и B}
  \label{fig:task3}
\end{wrapfigure}

$A$ и $B$ встречаются на интервале одного часа, который можно представить геометрически в виде рис. \ref{fig:task3}: 

\begin{itemize}
\item оси представляют время появления $A$ и $B$, соответственно,
\item область $S$, окрашенная светло-жёлтым, обозначает то множество временных точек, когда $A$ и $B$ могут встретиться: данная область сформирована, исходя из условия, что разница времён прихода $A$ и $B$ не может превышать 20 минут для того, чтобы встреча состоялась,
\item остальные две области за пределами $S$ обозначает то множество временных точек, когда A и B разминутся. 
\end{itemize}

Соответственно, отношение площади области S к общей площади (площади полной вероятности) и даст искомую вероятность встречи.

Для удобства счёта примем стороны квадрата за 6, тогда площади областей за пределами $S$ - это будут площади двух прямоугольных треугольников с катетами, равными 4.

\begin{align*}
p = \frac{\text{Площадь фигуры}\ S}{\text{Полная площадь}} = \frac{6 \times 6 - 2\frac{1}{2}4 \times 4}{6 \times 6} = \frac{20}{36} = \frac{5}{9}.
\end{align*}

\subsection{Сложная задача}

(не выполнена)

% ------------------------------------------------------------------------------
% Bibliography
% Is that possible to make it as per ГОСТ Р7.05-2008?
% ------------------------------------------------------------------------------
%\printbibliography[heading=bibintoc,title={References}]

\end{document}
