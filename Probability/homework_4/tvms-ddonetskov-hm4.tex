% !TEX TS-program = lualatex

% required for TexXstudio, an alternative is to change the default bibliography tool 
% in TeXstudio settings (Options > Configure TeXstudio > Build > Default Bibliography Tool)
% !BIB TS-program = biber

\documentclass[a4paper,11pt]{article}

\usepackage{geometry}
\geometry{left=20.00mm, right=15.00mm, top=20.00mm, bottom=20.00mm}

% this is to get rid of 'Overfull \hbox...' errors
\usepackage{microtype}

% ------------------------------------------------------------------------------
% Title
% ------------------------------------------------------------------------------

\title{\vspace{-1.5cm}Теория вероятностей и математическая статистика \\
Домашнее задание №4}
\author{Дмитрий Донецков (ddonetskov@gmail.com)}
\date{\today}

% ------------------------------------------------------------------------------
% LUA
% ------------------------------------------------------------------------------

% \usepackage{luacode}

% ------------------------------------------------------------------------------
% Graphics
% ------------------------------------------------------------------------------

\usepackage{graphicx}

% https://www.sharelatex.com/learn/Pgfplots_package
\usepackage{pgfplots}
\pgfplotsset{compat=1.15}\usepgfplotslibrary{fillbetween}

\usepackage{wrapfig}

% ------------------------------------------------------------------------------
% Figures
% ------------------------------------------------------------------------------

\usepackage{subcaption}         % subcaptions in figures

% ------------------------------------------------------------------------------
% Tables
% ------------------------------------------------------------------------------

\usepackage{multirow}           % spanning columns across multiple rows
\usepackage{makecell}           % allows different formats inside cells

\renewcommand\theadalign{bc}
\renewcommand\theadfont{\bfseries}
\renewcommand\theadgape{\Gape[4pt]}
\renewcommand\cellgape{\Gape[4pt]}

% ------------------------------------------------------------------------------
% Math (Additional Support)
% ------------------------------------------------------------------------------

\usepackage{amsmath,amsfonts,amssymb,amsthm,mathtools}     % AMS
\usepackage{cancel}             % four different modes of striking through
\usepackage{dsfont}
\usepackage{icomma}             % Smart comma: $0,2$ --- число, $0, 2$ --- перечисление
\usepackage{nicefrac}
\usepackage{physics}            % implementation of \abs and \norm

\DeclareMathOperator*{\D}{\mathbb{D}}   % the dispersion symbol
\DeclareMathOperator*{\E}{\mathbb{E}}   % the expectation symbol
\DeclareMathOperator*{\N}{\mathbb{N}}   % the set of natural numbers
\DeclareMathOperator*{\R}{\mathbb{R}}   % the set of real numbers
\DeclareMathOperator*{\Z}{\mathbb{Z}}   % the set of integers

% ------------------------------------------------------------------------------
% Russian Language (support thereof)
% ------------------------------------------------------------------------------

\usepackage[russian,english]{babel}	    % локализация и переносы

% ------------------------------------------------------------------------------
% Fonts 
% ------------------------------------------------------------------------------

\usepackage{fontspec}           % required to load Open Type, True Type fonts

\setmainfont{CMU Serif}
\setsansfont{CMU Sans Serif}
\setmonofont{CMU Typewriter Text}

%\setmainfont{Linux Libertine O} % Libertine covers Latin, Hebrew, Greek, and Russian
%\setmonofont{Courier New}

\usepackage{euscript}	          % Шрифт Евклид
\usepackage{mathrsfs}           % Красивый матшрифт

% ------------------------------------------------------------------------------
% Bibliography 
% ------------------------------------------------------------------------------

% Removed as it was not required.

% ------------------------------------------------------------------------------
% Bookmarking, citing, URL's
% ------------------------------------------------------------------------------

% hyperref usually needs to be loaded last
\usepackage{hyperref}
\usepackage{url}
% \usepackage[dvipsnames]{xcolor}

\hypersetup{
    colorlinks=true,
    linkcolor=blue,
    filecolor=red,      
    urlcolor=blue,
}

\urlstyle{same}

\begin{document}

\maketitle

\section{Задачи}

\subsection{Задача 1}

Рассмотрим $X \sim Bin(n, p)$, как сумму случайных величин Бернулли $X_i, i \in [1,n]$. Тогда производящая функция моментов для $X$:

\begin{align*}
\begin{cases}
  M_X(t) & = \prod_{i=1}^{n} M_{X_i}(t) = \prod_{i=1}^{n} \E[e^{tX_i}] = (\E[e^{tX_1}])^n, \\
  E[e^{tX_1}] & = e^{t \times 0}q + e^{t \times 1}p = q + pe^t, \\
  q & = 1 -p.
\end{cases}
\Rightarrow
\boxed{M_X(t) = (q + pe^t)^n}.
\end{align*}

Третий момент $\mu_3 = M^{(3)}_X(t)$:

\begin{align*}
M^{(1)}_X(t) & = [(q + pe^t)^n]' = pe^tn(q+pe^t)^{n-1},\\
  & \Rightarrow
  \mu_1 = M^{(1)}_X(0) = p1^nn(q+p1)^{n-1} = \boxed{pn}, \\
M^{(2)}_X(t) & = [M^{(1)}_X(t)]' = [pe^tn(q+pe^t)^{n-1}]' = pe^tn(q+pe^t)^{n-1} + pe^tn(n-1)pe^t(q+pe^t)^{n-2}\\
& = npe^t(q+pe^t)^{n-1} + n(n-1)[pe^t]^2((q+pe^t)^{n-2} \\
  & \Rightarrow
  \mu_2 = M^{(2)}_X(0) = np(1+p(n-1)) = \boxed{np(q+np)}.\\
\end{align*}

Нахождение $M^{(3)}_X(t)$ уже представляется делом громоздким, поэтому введём вспомогательную функцию:

\begin{align*}
z_t(k) = pe^t(q+pe^t)^k, k \in [0\cup\N].
\end{align*}

Тогда,

\begin{align*}
M^{(2)}_X(t) & = nz_t(k-1) + n(n-1)pe^tz_t(k-2), \\
M^{(3)}_X(t) & = nz'_t(k-1) + n(n-1)[pe^tz_t(k-2) + pe^tz'_t(k-2)] \\
 &= nz'_t(k-1) + n(n-1)pe^t[z_t(k-2) + z'_t(k-2)]. \\
\end{align*}

Так как для вычисления $\mu_3$ нам понадобится значение производящей функции в точке 0, то найдём значения $z_t(k-2)$, $z'_t(k-1)$, $z'_t(k-2)$ в нуле:

\begin{align*}
z_t(k)|_{t=0} & = p(q+p) = p, k \in [0\cup\N]. \\
z'_t(k) & = \frac{dz}{dt}[pe^t(q+pe^t)^k] = pe^t(q+pe^t)^k +[pe^t]^2k(q+pe^t)^{k-1} \\
& = pe^t(q+pe^t)^{k-1}(q+pe^t + pe^tk) = pe^t(q+pe^t)^{k-1}(q+(k+1)pe^t) \\
& = z_t(k)(q+(k+1)pe^t), \\
z'_t(k-1)|_{t=0} & = z_t(k-1)(q+(k-1+1)pe^t)|_{t=0} = p(q+kp), \\
z'_t(k-2)|_{t=0} & = z_t(k-2)(q+(k-2+1)pe^t)|_{t=0} = p(q+(k-1)p) = p(q + kp - p).
\end{align*}

Тогда

\begin{align*} 
\mu_3 & = M^{(3)}_X(0) = [nz'_t(k-1) + n(n-1)pe^t[z_t(k-2) + z'_t(k-2)]]|_{t=0} \\
& = n(p(q+np)) + n(n-1)p[p + p(q + np-p)] \\
& = np(q+np+(n-1)p(1+q+np-p)) \\
& = np(1-p+np+p(n-1)(2+np-2p)) \\
& = np(1-p+np+2np+n^2p^2-2np^2-2p-np^2+2p^2) \\
& = \boxed{np(1-3p+3np+n^2p^2-3np^2+2p^2)}.
\end{align*}

\subsection{Задача 2}

Пусть $X_1$ - появление 5 или 6 при одном выбросе кубика, тогда

\begin{itemize}
\item $X_1 \sim Bern(\frac{1}{3})$,
\item сумму реализаций $X_1$ при повторении эксперимента $n$ раз можно представить, как случайную величину $X$: $X \sim Bin(n, p)$, где $n = 120$, $p = \frac{1}{3}$,
\item $\E[X] = np = 120\frac{1}{3} = 40$,
\item $\D[X] = npq = 120\frac{1}{3}\frac{2}{3} \approx 26.6$,
\item интервал от 30 до 50 для реализации $X$ в серии выбросов даёт нам возможность определить максимальную разность, как $|X - \E[X]| = 10$.
\end{itemize}

Согласно неравенству Чебышева:

\begin{align*}
P(|X - \E[X]| & \leq 10) = 1 - P(|X - \E[X]| > 10), \\
P(|X - \E[X]| & > 10) \leq \frac{\D[X]}{10^2} = \frac{26.6}{100} \approx 0.26 \\
\Rightarrow P(|X - \E[X]| & \leq 10) > 1 - 0.26 = \boxed{0.74}.
\end{align*}

Обозначим через $S_n$ сумму реализаций $X_1$. Тогда, согласно центральной предельной теореме (ЦПТ):

\begin{align*}
S_n-n\mu \sim \frac{\sigma}{\sqrt{n}} \mathcal{N}(0, 1), n\mu = 40, \sigma = \sqrt{pq} = \frac{\sqrt{2}}{3}.
\end{align*}

\begin{align*}
P(30 \leq S_n \leq 50) & = \Phi_0(\frac{50 - 40}{\frac{\sqrt{2}}{3}\sqrt{120}}) - \Phi_0(\frac{30 - 40}{\frac{\sqrt{2}}{3}\sqrt{120}}) \\
& = \Phi_0(1.94) - \Phi_0(-1.94) = 0.97381 - 0.02619 \approx \boxed{0.95}.
\end{align*}

Здесь, стоит отметить, что вычисления по сути получаются такими же, как и при применении теоремы Муавра-Лапласа, т.к. последняя является частным случаем ЦПТ. Другими словами, для решения задачи мы могли бы перейти от рассмотрения применения ЦПТ для $X_1$ к рассмотрению применения теоремы Муавра-Лапласа для $X$.

\subsection{Задача 3}

Т.к. веса и их стандартные отклонения случайно выбранных арбуза и дыни - независимые случайные величины, то общий вес $X$ арбуза и дыни будет иметь среднее значение $\E[X] = 4 + 10 = 14$ (кг), а его стандартное отклонение будет $\sigma = \sqrt{\D[X]} \approx 4.12$ (кг), где $\D[X] = 1^2 + 4^2 = 17$ (кг).

\subsection{Задача 4}

Обозначим: $X$ - вес яблока, $X \sim \mathcal{N}(\mu = 95, \sigma^2 = 15^2)$, $S_n$ - общий вес $n$ яблок, $n$ = 10. Необходимо найти $P(S_n > 1000) = 1 - P(S_n < 1000)$. Тогда, согласно центральной предельной теореме:

\begin{align*}
P(S_n < 1000) & = P(\frac{S_n - n\mu}{\sigma\sqrt{n}} < \frac{1000 - n\mu}{\sigma\sqrt{n}}) 
= \Phi_0(\frac{1000 - 10*95}{15\sqrt{10}}) \approx 0.85314 \\
\Rightarrow 
P(S_n > 1000) & = 1 - 0.85314 = \boxed{0.14686}.
\end{align*}

\subsection{Сложная задача}

(не выполнена)

% ------------------------------------------------------------------------------
% Bibliography
% Is that possible to make it as per ГОСТ Р7.05-2008?
% ------------------------------------------------------------------------------
%\printbibliography[heading=bibintoc,title={References}]

\end{document}
