% !TEX TS-program = lualatex

% required for TexXstudio, an alternative is to change the default bibliography tool 
% in TeXstudio settings (Options > Configure TeXstudio > Build > Default Bibliography Tool)
% !BIB TS-program = biber

\documentclass[a4paper,11pt]{article}

\usepackage{geometry}
\geometry{left=20.00mm, right=15.00mm, top=20.00mm, bottom=20.00mm}

% this is to get rid of 'Overfull \hbox...' errors
\usepackage{microtype}

% ------------------------------------------------------------------------------
% Title
% ------------------------------------------------------------------------------

\title{\vspace{-1.5cm}Математический анализ и линейная алгебра \\
Домашнее задание №5}
\author{Дмитрий Донецков}
\date{\today}

% ------------------------------------------------------------------------------
% LUA
% ------------------------------------------------------------------------------

% \usepackage{luacode}

% ------------------------------------------------------------------------------
% Graphics
% ------------------------------------------------------------------------------

\usepackage{graphicx}

% https://www.sharelatex.com/learn/Pgfplots_package
\usepackage{pgfplots}
\pgfplotsset{compat=1.15}\usepgfplotslibrary{fillbetween}

\usepackage{wrapfig}

% ------------------------------------------------------------------------------
% Figures
% ------------------------------------------------------------------------------

\usepackage{subcaption}         % subcaptions in figures

% ------------------------------------------------------------------------------
% Tables
% ------------------------------------------------------------------------------

\usepackage{multirow}           % spanning columns across multiple rows
\usepackage{makecell}           % allows different formats inside cells

\renewcommand\theadalign{bc}
\renewcommand\theadfont{\bfseries}
\renewcommand\theadgape{\Gape[4pt]}
\renewcommand\cellgape{\Gape[4pt]}

% ------------------------------------------------------------------------------
% Math (Additional Support)
% ------------------------------------------------------------------------------

\usepackage{amsmath,amsfonts,amssymb,amsthm,mathtools}     % AMS
\usepackage{cancel}             % four different modes of striking through
\usepackage{dsfont}
\usepackage{icomma}             % Smart comma: $0,2$ --- число, $0, 2$ --- перечисление
%\usepackage{nicefrac}          % looks like xfrac is better maintained
\usepackage{physics}            % implementation of \abs and \norm
\usepackage{xfrac}

\DeclareMathOperator*{\D}{\mathbb{D}}   % the dispersion symbol
\DeclareMathOperator*{\E}{\mathbb{E}}   % the expectation symbol

\DeclareMathOperator*{\N}{\mathbb{N}}   % the set of natural numbers
\DeclareMathOperator*{\R}{\mathbb{R}}   % the set of real numbers
\DeclareMathOperator*{\Z}{\mathbb{Z}}   % the set of integers

% e = 2.71...
\newcommand{\e}{\mathrm{e}}

% sign of independence
% \vDash can also be used instead of \models
% \raisebox{}{} is required for vertical alignment
\newcommand{\independent}{\raisebox{0.05em}{\rotatebox[origin=c]{90}{$\models$}}}

%\newcommand\independent{\protect\mathpalette{\protect\independenT}{\perp}}
%\def\independenT#1#2{\mathrel{\rlap{$#1#2$}\mkern2mu{#1#2}}}

% adding options to the amsmath's matrix environments as per
% http://texblog.net/latex-archive/maths/amsmath-matrix/
\makeatletter
\renewcommand*\env@matrix[1][*\c@MaxMatrixCols c]{%
	\hskip -\arraycolsep
	\let\@ifnextchar\new@ifnextchar
	\array{#1}}
\makeatother

% ------------------------------------------------------------------------------
% Russian Language (support thereof)
% ------------------------------------------------------------------------------

\usepackage[russian,english]{babel}	    % локализация и переносы

% ------------------------------------------------------------------------------
% Fonts 
% ------------------------------------------------------------------------------

\usepackage{fontspec}           % required to load Open Type, True Type fonts

\setmainfont{CMU Serif}
\setsansfont{CMU Sans Serif}
\setmonofont{CMU Typewriter Text}

%\setmainfont{Linux Libertine O} % Libertine covers Latin, Hebrew, Greek, and Russian
%\setmonofont{Courier New}

\usepackage{euscript}	          % Шрифт Евклид
\usepackage{mathrsfs}           % Красивый матшрифт

% ------------------------------------------------------------------------------
% Bibliography 
% ------------------------------------------------------------------------------

% Removed as it was not required.

% ------------------------------------------------------------------------------
% Bookmarking, citing, URL's
% ------------------------------------------------------------------------------

% hyperref usually needs to be loaded last
\usepackage{hyperref}
\usepackage{url}
% \usepackage[dvipsnames]{xcolor}

\hypersetup{
    colorlinks=true,
    linkcolor=blue,
    filecolor=red,      
    urlcolor=blue,
}

\urlstyle{same}

% ------------------------------------------------------------------------------
% Various
% ------------------------------------------------------------------------------
\usepackage{listings}
\lstset{showstringspaces=false}

% ------------------------------------------------------------------------------
% Document
% ------------------------------------------------------------------------------

\begin{document}

\maketitle

\section{Задачи}

\subsection{Задача 1}

Данная задача была решена ранее, как задача 4 задания 4.

\subsection{Задача 2}

В общем случае, множество многочленов $S = P[n]$ при всех $n$ меньше определённого числа удовлетворяет требованиям к линейному пространству, поэтому оно является линейным пространством, размерностью $n+1$. Примером базиса для них является вектор $(1, x, x^2, \dots, x^n)$.

Посмотрим, как на это влияют дополнительные условия:

a) Если каждый многочлен множества $S$ равен нулю в единице, то ...

b) Если каждый многочлен множества $S$ равен пяти в единице, то ... .

\subsection{Задача 3}

Заданная система линейных уравнений не имеет однозначного решения, т.к. содержит четыре неизвестных переменных при трёх уравнениях. Попробуем прийти к общему решению методом Гаусса:

\begin{align*}
\begin{bmatrix}[rrrr|r]
  -6 &  9 &  3 &  2 &  4 \\
  -2 &  3 &  5 &  2 &  2 \\
  -2 &  6 &  4 &  3 &  3
\end{bmatrix}
& \begin{matrix}[c] ~ \\ \xrightarrow{R_1-3R_2} \\ \xrightarrow{R_1-3R_3} \end{matrix}
\begin{bmatrix}[rrrr|r]
  -6 &   9 &   3 &   2 &   4 \\
   0 &   0 & -12 &  -4 &  -2 \\
   0 &  -9 &  -9 &  -7 &  -5
\end{bmatrix}
\begin{matrix}[c] \xrightarrow{R_1+R_3} \\ ~ \\ ~ \end{matrix}
\begin{bmatrix}[rrrr|r]
  -6 &   0 &  -6 &  -5 &  -1 \\
   0 &   0 & -12 &  -4 &  -2 \\
   0 &  -9 &  -9 &  -7 &  -5
\end{bmatrix}
\\
& \begin{matrix}[c] \xrightarrow{2R_1-R_2} \\ ~ \\ ~ \end{matrix}
\begin{bmatrix}[rrrr|r]
 -12 &   0 &   0 &  -6 &   0 \\
   0 &   0 & -12 &  -4 &  -2 \\
   0 &  -9 &  -9 &  -7 &  -5
\end{bmatrix}
\begin{matrix}[c] ~ \\ \xrightarrow{R_2 \leftrightarrow R_3} \\ ~ \end{matrix}
\begin{bmatrix}[rrrr|r]
 -12 &   0 &   0 &  -6 &   0 \\
   0 &  -9 &  -9 &  -7 &  -5 \\
   0 &   0 & -12 &  -4 &  -2
\end{bmatrix}
\\
& \begin{matrix}[c] ~ \\ \xrightarrow{R_2-\frac{9}{12}R_3} \\ ~ \end{matrix}
\begin{bmatrix}[rrrr|r]
 -12 &   0 &   0 &  -6 &   0 \\
   0 &  -9 &   0 &  -4 &  -\sfrac{7}{2} \\
   0 &   0 & -12 &  -4 &  -2
\end{bmatrix}
\begin{matrix}[c] ~ \\ ~\xrightarrow{} \\ ~ \end{matrix}
\begin{bmatrix}[rrrr|r]
   2 &   0 &   0 &   1 &   0 \\
   0 &  18 &   0 &   8 &   7 \\
   0 &   0 &   6 &   2 &   1
\end{bmatrix}.
\end{align*}

Тогда, бесконечное множество решений заданной системы линейных уравнений определяется значением $x_4$:

\begin{align*}
\begin{cases}
  x_1 & = -\frac{1}{2}x_4, \\
  x_2 & = \frac{7-8x_4}{18}, \\
  x_3 & = \frac{1 - 2x_4}{6}.
\end{cases}
\quad \Leftrightarrow \quad
\begin{pmatrix}
  x_1 \\
  x_2 \\
  x_3
\end{pmatrix}
=
\begin{pmatrix}
  0 \\
  \sfrac{7}{18} \\
  \sfrac{1}{6}
\end{pmatrix}
-
\begin{pmatrix}
  \sfrac{1}{2} \\
  \sfrac{4}{9} \\
  \sfrac{1}{3}
\end{pmatrix}
x_4.
\end{align*}

\subsection{Задача 4}

Следующие комбинации векторов будут образовывать базис в $\R^2$, т.к. будут являться линейно-независимыми:

\begin{itemize}
  \item $a_1$ и $a_2$,
  \item $a_1$ и любой из векторов $a_3$, $a_4$, $a_5$,
  \item $a_2$ и любой из векторов $a_3$, $a_4$, $a_5$.    
\end{itemize}

\subsection{Задача 5}

Из трех данных векторов посредством составления всех их линейных комбинаций можно получить линейное пространство векторов на множестве $\R^3$. Для линейного пространства на множестве $\R^3$ можно взять базис $e_1 = (1, 0, 0), e_2 = (0, 1, 0), e_3 = (0, 0, 1)$ размерностью 3.

\subsection{Задача 6}

а) Сумма и произведения существуют и равны:

\begin{align*}
A + B = 
\begin{bmatrix}[rr]
 1 &  4 \\
-5 &  2 
\end{bmatrix}
+
\begin{bmatrix}[rr]
 3 &  8 \\
 1 &  2 
\end{bmatrix}
=
\begin{bmatrix}[rr]
 4 & 12 \\
-4 &  4
\end{bmatrix}
\end{align*}

\begin{align*}
A \times B = 
\begin{bmatrix}[rr]
 1 &  4 \\
-5 &  2 
\end{bmatrix}
\times
\begin{bmatrix}[rr]
 3 &  8 \\
 1 &  2 
\end{bmatrix}
=
\begin{bmatrix}[rr]
  7 &  16 \\
-13 & -36
\end{bmatrix}.
\end{align*}

б) Сумма матриц $A$, $B$ не определена, т.к. у них разная размерность; произведение - существует и равно:

\begin{align*}
A \times B = 
\begin{bmatrix}[rrr]
  0 & 4 & 2 \\
 -3 & 5 & 0 
\end{bmatrix}
\times
\begin{bmatrix}[rr]
  7 &  3 \\
  5 & -2 \\
 12 &  0
\end{bmatrix}
=
\begin{bmatrix}[rr]
 44 &  -8 \\
  4 & -19 
\end{bmatrix}.
\end{align*}

\subsection{Задача 7}

$\rank{A} = 2$, т.к. в результате преобразований по методу Гаусса в ней остаются две ненулевых строки.

\begin{align*}
A =
\begin{bmatrix}[rrr]
   5 &  3 &  8 \\
   4 &  3 &  1 \\
   3 &  2 &  3
\end{bmatrix}
& \begin{matrix}[c] ~ \\ \xrightarrow{R_1-R_2} \\ ~ \end{matrix}
\begin{bmatrix}[rrr]
   5 &  3 &  8 \\
   1 &  0 &  7 \\
   3 &  2 &  3
\end{bmatrix}
\begin{matrix}[c] ~ \\ ~ \\ \xrightarrow{R_3-3R_2} \end{matrix}
\begin{bmatrix}[rrr]
   5 &  3 &   8 \\
   1 &  0 &   7 \\
   0 &  2 & -18
\end{bmatrix}
\\
& \begin{matrix}[c] \xrightarrow{R_1-5R_2} \\ ~ \\ ~ \end{matrix}
\begin{bmatrix}[rrr]
   0 &  3 & -27 \\
   1 &  0 &   7 \\
   0 &  2 & -18
\end{bmatrix}
\begin{matrix}[c] ~ \\ ~ \\ \xrightarrow{R_3-\frac{2}{3}R_1} \end{matrix}
\begin{bmatrix}[rrr]
   0 &  3 & -27 \\
   1 &  0 &   7 \\
   0 &  0 &   0
\end{bmatrix}.
\end{align*}

\subsection{Задача 8}

\begin{align*}
A^2 & =
\begin{bmatrix}[rrr]
  16 &  -8 &   1 \\
   0 &  16 &  -8 \\
   0 &   0 &  16
\end{bmatrix}
\quad
A^3 = A^2 A =
\begin{bmatrix}[rrr]
 -64 &  48 & -12 \\
   0 & -64 &  48 \\
   0 &   0 & -64
\end{bmatrix}
\quad
A^4 = A^3 A =
\begin{bmatrix}[rrr]
 256 &-256 &  96 \\
   0 & 256 &-256 \\
   0 &   0 & 256
\end{bmatrix}
\\
& \Rightarrow
A^n = A^{n-1}A =
\begin{bmatrix}[rrr]
  (-4)^n &      * & * \\
       0 & (-4)^n & * \\
       0 &      0 & (-4)^n
\end{bmatrix}.
\end{align*}

* - данные элементы не получается выразить сразу же через элементы исходной матрицы $A$.

\subsection{Задача 9}

a) 

\begin{align*}
(A|E) & = 
\begin{bmatrix}[rr|rr]
   3 &  7 &  1 &  0 \\
   4 &  9 &  0 &  1
\end{bmatrix}
\begin{matrix}[c] ~ \\ \xrightarrow{R_2-\sfrac{4}{3}R_1} \end{matrix}
\begin{bmatrix}[rr|rr]
   3 &             7 &  1             &  0 \\
   0 & -\sfrac{1}{3} &  -\sfrac{4}{3} &  1
\end{bmatrix}
\begin{matrix}[c] \xrightarrow{R_1+21R_2} \\ ~ \end{matrix}
\begin{bmatrix}[rr|rr]
   3 &             0 &  -27           & 21 \\
   0 & -\sfrac{1}{3} &  -\sfrac{4}{3} &  1
\end{bmatrix}
\begin{matrix}[c] \xrightarrow{\divisionsymbol 3} \\ \xrightarrow{\times -3} \end{matrix}
\begin{bmatrix}[rr|rr]
   1 & 0 & -9 &  7 \\
   0 & 1 &  4 & -3
\end{bmatrix}
\\
& \Rightarrow
A^{-1} = 
\begin{bmatrix}[rr]
   -9 &  7 \\
    4 & -3
\end{bmatrix}.
\end{align*}

b) 

\begin{align*}
(A|E) & = 
\begin{bmatrix}[rr|rr]
   a & b &  1 &  0 \\
   c & d &  0 &  1
\end{bmatrix}
\begin{matrix}[c] ~ \\ \xrightarrow{R_2-\sfrac{c}{a}R_1} \end{matrix}
\begin{bmatrix}[rr|rr]
   a & b                &  1             &  0 \\
   0 & d - b\,\sfrac{c}{a}  &  -\sfrac{c}{a} &  1
\end{bmatrix}
\xrightarrow{R_1-\frac{ab}{ad-bc}R_2} 
\begin{bmatrix}[rr|rr]
   a & 0                    &  \frac{ad}{ad-bc}  &  -\frac{ab}{ad-bc} \\
   0 & d - b\,\sfrac{c}{a}  &  -\sfrac{c}{a}            &  1
\end{bmatrix}
\\
& \Rightarrow
\begin{bmatrix}[rr|rr]
   1 & 0  &  \frac{d}{ad-bc}  &  -\frac{b}{ad-bc} \\
   0 & 1  & -\frac{c}{ad-bc}  &  \frac{a}{ad-bc}
\end{bmatrix}
\\
& \Rightarrow
A^{-1} = 
\frac{1}{ad-bc}
\begin{bmatrix}[rr]
   d & -b \\
  -c &  a
\end{bmatrix}.
\end{align*}

c) Не выполнено.

d) Не выполнено.

\subsection{Задача 10}

Обозначим через $A$ исходную матрицу $A$, через $A'$ - изменённую матрицу.\\

a) $A'$ - это матрица $A$, в которой строки $i,j$ были переставлены местами. Такую операцию перестановки строк в матричном виде можно выразить через операцию умножения:

\begin{align*}
  A' = E'A \, ,
\end{align*}

где $E'$ - это единичная матрица $I$, в которой были переставлены строки $i,j$.

Тогда для ответа на заданный вопрос нам от данного выражения следует прийти к выражению с обратными матрицами для $A, A'$, чтобы увидеть, как они будут связаны между собой:

\begin{align}
  \label{eq:t10.1}
  (A')^{-1} = (E'A)^{-1} = A^{-1}(E')^{-1} = A^{-1}E'.
\end{align}

Т.е. выражение \ref{eq:t10.1} показывает, что в обратной матрице для $A'$ местами будут поменяны столбцы $i,j$ относительно обратной матрицы для $A$, т.к. $A^{-1}E'$ - это формула перестановки столбцов.

N.B. Я не смог доказать выше, что $(E')^{-1} = E'$, но эмпирические вычисления показали справедливость такого перехода.\\

b) Не выполнено.

\subsection{Задача 11}

a) $\det A = 4 \times 9 - 8 \times 1 = 28$.

b) $\det B = 12 + 0 + 36 - 8 - 90 - 0 = -50$.

c) Не выполнено.

\end{document}
