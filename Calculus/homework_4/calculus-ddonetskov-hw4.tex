% !TEX TS-program = lualatex

% required for TexXstudio, an alternative is to change the default bibliography tool 
% in TeXstudio settings (Options > Configure TeXstudio > Build > Default Bibliography Tool)
% !BIB TS-program = biber

\documentclass[a4paper,11pt]{article}

\usepackage{geometry}
\geometry{left=20.00mm, right=15.00mm, top=20.00mm, bottom=20.00mm}

% this is to get rid of 'Overfull \hbox...' errors
\usepackage{microtype}

% ------------------------------------------------------------------------------
% Title
% ------------------------------------------------------------------------------

\title{\vspace{-1.5cm}Математический анализ и линейная алгебра \\
Домашнее задание №3}
\author{Дмитрий Донецков (ddonetskov@gmail.com)}
\date{\today}

% ------------------------------------------------------------------------------
% LUA
% ------------------------------------------------------------------------------

% \usepackage{luacode}

% ------------------------------------------------------------------------------
% Graphics
% ------------------------------------------------------------------------------

\usepackage{graphicx}

% https://www.sharelatex.com/learn/Pgfplots_package
\usepackage{pgfplots}
\pgfplotsset{compat=1.15}\usepgfplotslibrary{fillbetween}

\usepackage{wrapfig}

% ------------------------------------------------------------------------------
% Figures
% ------------------------------------------------------------------------------

\usepackage{subcaption}         % subcaptions in figures

% ------------------------------------------------------------------------------
% Tables
% ------------------------------------------------------------------------------

\usepackage{multirow}           % spanning columns across multiple rows
\usepackage{makecell}           % allows different formats inside cells

\renewcommand\theadalign{bc}
\renewcommand\theadfont{\bfseries}
\renewcommand\theadgape{\Gape[4pt]}
\renewcommand\cellgape{\Gape[4pt]}

% ------------------------------------------------------------------------------
% Math (Additional Support)
% ------------------------------------------------------------------------------

\usepackage{amsmath,amsfonts,amssymb,amsthm,mathtools}     % AMS
\usepackage{cancel}             % four different modes of striking through
\usepackage{dsfont}
\usepackage{icomma}             % Smart comma: $0,2$ --- число, $0, 2$ --- перечисление
%\usepackage{nicefrac}          % looks like xfrac is better maintained
\usepackage{physics}            % implementation of \abs and \norm
\usepackage{xfrac}

\DeclareMathOperator*{\D}{\mathbb{D}}   % the dispersion symbol
\DeclareMathOperator*{\E}{\mathbb{E}}   % the expectation symbol

\DeclareMathOperator*{\N}{\mathbb{N}}   % the set of natural numbers
\DeclareMathOperator*{\R}{\mathbb{R}}   % the set of real numbers
\DeclareMathOperator*{\Z}{\mathbb{Z}}   % the set of integers

% e = 2.71...
\newcommand{\e}{\mathrm{e}}

% sign of independence
% \vDash can also be used instead of \models
% \raisebox{}{} is required for vertical alignment
\newcommand{\independent}{\raisebox{0.05em}{\rotatebox[origin=c]{90}{$\models$}}}

%\newcommand\independent{\protect\mathpalette{\protect\independenT}{\perp}}
%\def\independenT#1#2{\mathrel{\rlap{$#1#2$}\mkern2mu{#1#2}}}

% ------------------------------------------------------------------------------
% Russian Language (support thereof)
% ------------------------------------------------------------------------------

\usepackage[russian,english]{babel}	    % локализация и переносы

% ------------------------------------------------------------------------------
% Fonts 
% ------------------------------------------------------------------------------

\usepackage{fontspec}           % required to load Open Type, True Type fonts

\setmainfont{CMU Serif}
\setsansfont{CMU Sans Serif}
\setmonofont{CMU Typewriter Text}

%\setmainfont{Linux Libertine O} % Libertine covers Latin, Hebrew, Greek, and Russian
%\setmonofont{Courier New}

\usepackage{euscript}	          % Шрифт Евклид
\usepackage{mathrsfs}           % Красивый матшрифт

% ------------------------------------------------------------------------------
% Bibliography 
% ------------------------------------------------------------------------------

% Removed as it was not required.

% ------------------------------------------------------------------------------
% Bookmarking, citing, URL's
% ------------------------------------------------------------------------------

% hyperref usually needs to be loaded last
\usepackage{hyperref}
\usepackage{url}
% \usepackage[dvipsnames]{xcolor}

\hypersetup{
    colorlinks=true,
    linkcolor=blue,
    filecolor=red,      
    urlcolor=blue,
}

\urlstyle{same}

% ------------------------------------------------------------------------------
% Various
% ------------------------------------------------------------------------------
\usepackage{listings}
\lstset{showstringspaces=false}

% ------------------------------------------------------------------------------
% Document
% ------------------------------------------------------------------------------

\begin{document}

\maketitle

\section{Задачи}

\subsection{Задача 1}

Данная задача была решена ранее, как задача 5 задания 3.

\subsection{Задача 2}

Найдём частные производные $f(x,y) = y \e^{xy}$:

\begin{align*}
f_x & = y^2 \e^{xy} \\
f_y & = \e^{xy} + y(x\e^{xy}) = \e^{xy}(1+xy).
\end{align*}

Отсюда, составим уравнение касательной плоскости в точке $(a,b)$:

\begin{align*}
z & = f(a, b) + f_x(a, b)(x-a) + f_y(a, b)(y-b) \\
  & = b \e^{ab} + b^2 \e^{ab} (x-a) + \e^{ab}(1+ab)(y-b) \\
  & = \e^{ab} (b + b^2 (x-a) + (1+ab)(y-b)).
\end{align*}

Тогда, мы сможем воспользоваться уравнением данной касательной плоскости в точке $(a,b)$, как линейной аппроксимацией значения функции в окрестности данной точки. Так, для заданной точки $(x,y)  = (-0.1, 1.1)$:

\begin{align*}
f(x, y) \bigg \rvert_{(x, y) = (-0.1, 1.1)} & \approx \e^{ab} (b + b^2 (x-a) + (1+ab)(y-b)) \bigg \rvert_{(a,b) = (0,1), (x, y) = (-0.1, 1.1)} \\
 & = 1 (1 + 1^2(x-0) + 1(y-1)) \rvert_{(x, y) = (-0.1, 1.1)} = 1 - 0.1 + (1.1 -1) = 1.
\end{align*}

Т.е. приближённое значение данной функции в точке $(x,y) = (-0.1, 1.1)$ равно 1. Справочно, более точное значение функции в заданной точке равно 0.9854175488261812, следовательно, погрешность вычисления составила примерно 1.5\%.

\subsection{Задача 3}

Вычислим требуемые частные производные для $f(x, y) = \sin (x+y) - \cos (x^2)$:

\begin{minipage}{0.4\linewidth}
\begin{align*}
f_x & = \cos (x+y) + 2x \sin (x^2), \\
f_y & = \cos (x+y),
\end{align*}
\end{minipage}
\begin{minipage}{0.6\linewidth}
	\begin{align*}
f_{xx} & = -\sin (x+y)  + 2 \sin (x^2) + 2x 2x \cos (x^2), \\
f_{yy} & = -\sin (x+y), \\
f_{xy} & = -\sin (x+y).
\end{align*}
\end{minipage}

Тогда ряд Тейлора до второго порядка включительно для указанной функции в точке (0, 0) будет следующим:

\begin{align*}
f(x, y) & = f(0, 0) + f_x(0, 0)x + f_y(0, 0)y + \frac{1}{2!}
  (f_{xx}(0,0)x^2 + 2f_{xy}(0,0)xy + f_{yy}(0,0)y^2) + R_n \\
   & = -1 + x + y + \frac{1}{2}(0) + R_n = x + y - 1 + R_n.
\end{align*}

\subsection{Задача 4}

Область $\R^2$, ограниченная $x = 1, y^2 = 2x$, - это область с координатами $(x, y) \in [0, 1] \times [-\sqrt{2x}, \sqrt{2x}]$.

Тогда заданный двойной интеграл равен:

\begin{align*}
\iint \limits_{\R^2:\ x = 1, y^2 = 2x} xy^2\,dx\,dy 
& = \int_{0}^{1} \int_{-\sqrt{2x}}^{\sqrt{2x}} xy^2\,dy\,dx \\
& = \int_{0}^{1} \frac{1}{3}xy^3 \bigg \rvert_{-\sqrt{2x}}^{\sqrt{2x}} \,dx 
  = \int_{0}^{1} \frac{4\sqrt{2}}{3}x^{\sfrac{5}{2}} \,dx \\
& = \frac{4\sqrt{2}}{3} \frac{2}{7} x^{\sfrac{7}{2}} \bigg \rvert_{0}^{1} = \frac{8\sqrt{2}}{21}.
\end{align*}

\end{document}
